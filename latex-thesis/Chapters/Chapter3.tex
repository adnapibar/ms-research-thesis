% Chapter 3

\chapter{SCATS Volume Data} % Main chapter title

% For referencing the chapter elsewhere, use \ref{Chapter3}
\label{Chapter3}

% This is for the header on each page - perhaps a shortened title
\lhead{Chapter 3. \emph{SCATS Volume Data}}

%----------------------------------------------------------------------------------------
% Quotation
``There is no order in the world around us, we must adapt ourselves to the requirements of chaos
instead."

\begin{flushright}
Kurt Vonnegut, \textit{Breakfast of Champions} (1973)
\end{flushright}

%---------------------------------------------------------------------------------------------------
%	CONTENT
%   Reference - http://www.scats.com.au/files/an_introduction_to_scats_6.pdf
%---------------------------------------------------------------------------------------------------
\section{Introduction}
SCATS(Sydney Coordinated Adaptive Traffic System) is an adaptive traffic control system. It was
developed by the Department of Main Roads in the 1970's. SCATS operates in real-time by adjusting
signal timings in response to changes in traffic demand and road capacity. All major and minor
cities in Australia and New Zealand use SCATS. Few other cities around the world such as Hong
Kong, Kuala Lumpur, Sanghai and Singapore also have adopted SCATS over other adaptive traffic
control system. In Melbourne and surrounding cities, SCATS controls more than 3,900 sets of traffic
signals

There are three main parameters that SCATS user to achieve traffic signal coordination:
\begin{itemize}
\item[\tiny{$\blacksquare$}] Cycle time: The total time of all signal sequences in a cycle
\item[\tiny{$\blacksquare$}] Phase split: The proportion of the cycle time allocated to each phase
\item[\tiny{$\blacksquare$}] Offset: The time relationship between the starting and finishing of
the green phases of succesive sets of signals within a coordinated system
\end{itemize}

The desicion making of the SCATS system occurs at two levels - \emph{strategic} and \emph{tactical}.


\section{Traffic flow volume data}

% Describe the traffic flow

\subsection{Data acquisition}
Traffic loop detectors are embedded in the raod pavement and located in each lane near the stop
line at traffic intersections. These detectors collect traffic volume and the time it takes a
vehicle to clear the loop. A schematic diagram of a loop detector is shown in fig \ref{fig:loopDetector}

\begin{figure}[htbp]
  \centering
    \includegraphics[width=0.7\textwidth,height=0.7\textheight,keepaspectratio]{Figures/loop-detector.pdf}
    \rule{35em}{0.5pt}
  \caption[A vehicle loop detector system]{An inductive loop detector system used for detecting the presence
  of a vehicle.}
  \label{fig:loopDetector}
\end{figure}


\subsection{Measurement errors}
There are various factors that contribute to the measurement errors. This can be the fault of the
vehicle loop detection system or at the traffic control center.

\subsubsection{Missing data}
One of the major difficulties with traffic sensor data is missing data, that can be caused by
several factors.

% table showing count of missing data
% plot of missing data

\subsubsection{Unreliable data}

\section{Analysis of the traffic volume data}

% Trends - daily, weekly, weekdays, weekends, special events
% Seasonality
% Spatial relationships

In this section, we present analysis on the traffic volume data at a homogenous segment.
For this purpose we will use both the south and north bound traffic data at Nicholson
street (north of Melbourne's CBD)between Gertrude street and Victoria Parade during the period
from 01/01/2008 to 25/07/2013. Traffic volume was aggregated on a 15 minutes interval making a total
of 195168 observations.


Figure \ref{fig:AverageTrafficVolume} shows the daily, weekly, monthly and yearly average traffic
volume at a site location.Figure \ref{fig:TypicalDayTraffic} shows how a typical day of the week
on average looks like at a homogeneous link.

\begin{figure}[h]
    \centering
    \subfloat[Daily][Daily]{
    \includegraphics[width=0.4\textwidth]{Figures/averages-daily.pdf}
    \label{fig:AverageDaily}}
    \qquad
    \subfloat[Weekly][Weekly]{
    \includegraphics[width=0.4\textwidth]{Figures/averages-weekly.pdf}
    \label{fig:AverageWeekly}}

    \subfloat[Monthly][Monthly]{
    \includegraphics[width=0.4\textwidth]{Figures/averages-monthly.pdf}
    \label{fig:AverageMonthly}}
    \qquad
    \subfloat[Yearly][Yearly]{
    \includegraphics[width=0.4\textwidth]{Figures/averages-yearly.pdf}
    \label{fig:AverageYearly}}

    \caption[Average Traffic Volume]{(a) daily, (b) weekly, (c) monthly and (d) yearly average of
    traffic volume (15 mins interval) at a site location from the period 01/01/2008 to 26/07/2013}
   \label{fig:AverageTrafficVolume}
\end{figure}

\begin{figure}[h]
    \centering
    \subfloat[Monday][Monday]{
    \includegraphics[width=0.4\textwidth]{Figures/typical-Monday.pdf}
    \label{fig:typicalMonday}}
    \qquad
    \subfloat[Tuesday][Tuesday]{
    \includegraphics[width=0.4\textwidth]{Figures/typical-Tuesday.pdf}
    \label{fig:typicalTuesday}}

    \subfloat[Wednesday][Wednesday]{
    \includegraphics[width=0.4\textwidth]{Figures/typical-Wednesday.pdf}
    \label{fig:typicalWednesday}}
    \qquad
    \subfloat[Thursday][Thursday]{
    \includegraphics[width=0.4\textwidth]{Figures/typical-Thursday.pdf}
    \label{fig:typicalThursday}}

    \subfloat[Friday][Friday]{
    \includegraphics[width=0.4\textwidth]{Figures/typical-Friday.pdf}
    \label{fig:typicalFriday}}
    \qquad
    \subfloat[Saturday][Saturday]{
    \includegraphics[width=0.4\textwidth]{Figures/typical-Saturday.pdf}
    \label{fig:typicalSaturday}}

    \caption[Average traffic grouped by every day of the week]{Average traffic grouped by every
    day of the week.}
    \label{fig:TypicalDayTraffic}
\end{figure}

\begin{figure}[h]
    \centering
    \subfloat[ACF][ACF]{
    \includegraphics[width=0.4\textwidth]{Figures/acf.pdf}
    \label{fig:acf}}
    \qquad
    \subfloat[PACF][PACF]{
    \includegraphics[width=0.4\textwidth]{Figures/pacf.pdf}
    \label{fig:pacf}}

   \caption[Plots of ACF and PACF]{Plots of the autocorrelation and partial autocorrelation
   functions}
   \label{fig:acfPacf}
\end{figure}