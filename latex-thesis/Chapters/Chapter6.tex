% Chapter 6

\chapter{Conclusions and Future Directions} % Main chapter title

\label{Chapter6} % For referencing the chapter elsewhere, use \ref{Chapter6}

% This is for the header on each page - perhaps a shortened title
\lhead{Chapter 6. \emph{Conclusions and Future Directions}}

% Quotation
``Everything should be made as simple as possible but not simpler."

\begin{flushright}
Albert Einstein
\end{flushright}

%---------------------------------------------------------------------------------------------------
%	CONTENT
%---------------------------------------------------------------------------------------------------

\section{Conclusions}

The objective of this work was to use deep neural networks in the field of short term traffic
prediction. Deep neural networks have been used recently to solve various complex tasks in the
field of computer vision, speech recognition and natural language modelling, however they can be
used to solve other problems. We presented the state of the art of various traffic prediction
methods and reviewed some of the works in short term traffic prediction. Most of the available
methods can be categorised into four groups - naive, parametric, non-parametric and hybrid
methodologies. We presented a high level theoretical comparison of these methods, however a fair
comparison of these methods using the existing literature is difficult. This is mainly due to the
significant differences in the experimental studies presented in these works, which used datasets
from different locations and time.

We analysed the traffic volume data collected by VicRoads using vehicle loop detectors. A suitable
traffic region was selected based on the data of number of missing days count. We presented the
variations in daily traffic flow at a homogeneous road segment in this selected region. We found
that the traffic variations on a day is largely influenced by the location and day of the week.
We also analysed the influence of public events on traffic flow data. Both the temporal and
spatial relations in traffic flow data were analysed. The temporal characteristics are found in the
traffic flow data by using autocorrelations. The presence of spatial relations of traffic flow
data from current location with upstream locations and adjacent locations were detected using the
cross correlations.

We also briefly introduced the deep neural networks especially the Long Short Term Memory recurrent
neural networks. The architecture and training of LSTM models were presented. We used the traffic
flow volume data for our experiments and model evaluations. A range of existing methods were used
for a broad comparison.

\section{Future directions}
The application of deep neural networks to solve various problems are still at infancy. We believe in
coming years their use will increase to solve a range of other problems.