% Chapter 6

\chapter{Conclusions and Future Directions} % Main chapter title

\label{Chapter6} % For referencing the chapter elsewhere, use \ref{Chapter6}

% This is for the header on each page - perhaps a shortened title
\lhead{Chapter 6. \emph{Conclusions and Future Directions}}

% Quotation
``Everything should be made as simple as possible but not simpler."

\begin{flushright}
Albert Einstein
\end{flushright}

%---------------------------------------------------------------------------------------------------
%	CONTENT
%---------------------------------------------------------------------------------------------------

\section{Conclusions}

The objective of this work was two fold; first to find out whether we can use the large amount of
available traffic volume data for short term traffic flow predictions. By reviewing existing literature,
we understood that the data driven algorithms are capable of handling large amount of traffic volume
data. Secondly we wanted to find out whether the application of deep neural networks can use the
spatio-temporal relations in such data to provide better prediction accuracies. We have seen that
deep neural networks have been used recently to solve various complex tasks in the field of computer
vision, speech recognition and natural language modelling, however can they be used to solve other problems?

In chapter \ref{Chapter2}, We presented the state of the art of various traffic prediction methods
and reviewed existing literature in short term traffic prediction. The complex and highly non-linear
nature of traffic volume data make it difficult to accurately predict the traffic in the short term.
We found that most of the available methods can be categorised into four groups - naive, parametric,
non-parametric and hybrid methodologies.
We presented a high level theoretical comparison of these methods, however a fair comparison of
these methods using the existing literature is difficult. This is mainly due to the significant
differences in the experimental studies presented in these works, which used datasets from different
locations and time.

We analysed the traffic volume data collected by VicRoads using vehicle loop detectors. A suitable
traffic region was selected based on the number of missing days. We presented the
variations in daily traffic flow at a homogeneous road segment in this selected region. We found
that the traffic variations on a day is largely influenced by the location and day of the week.
We also analysed the influence of public events on traffic flow data. Both the temporal and
spatial relations in traffic flow data were analysed. The temporal characteristics are found in the
traffic flow data by using autocorrelations. The presence of spatial relations of traffic flow
data from current location with upstream locations and adjacent locations were detected using the
cross correlations.

We briefly introduced the deep neural networks especially the Long Short Term Memory networks.
We conducted the experiments to find out the performance of deep neural networks in predicting traffic
flow. For comparison purpose we used a range of existing methods to set up benchmarks. The experiments
were conducted in both univariate setting, where only data form one location was used for modelling, and
multivariate setting where data from multiple locations were used for modelling the spatio-temporal
characteristics. We found out from the results that the data driven algorithms outperformed the parametric
methods. The application of deep neural networks showed promising results in both univariate and
multivariate settings. The results of the GRU and LSTM networks were comparatively similar but better
than the simple RNN networks. The GRU (multivariate) networks had the best accuracy in predicting 15-minutes traffic
volume prediction, slightly better than LSTM (multivariate), while LSTM (multivariate) networks had
best accuracies for 30-minutes and 45-minutes predictions. From these results we can positively
conclude our second objective of this research.

\section{Future directions}
The application of deep neural networks to solve various problems are still at infancy. We believe in
coming years their use will increase to solve a range of other problems. This work may provide a basis
for further experiments of deep neural networks in short term traffic predictions. The network topologies
used in this work can be modified for achieving better results. Implementation of such a network at much
larger scale that spans hundreds of locations is computationally very expensive and may not generalise
well as the spatial information will get lost in such large scale. In this work we manually handpicked
a region and used traffic data from nearby locations. This can be automated by using clustering
to partition the traffic volume data of the entire transportation network into several clusters
based on strong spatial correlations. Then each of these clusters can be modelled using a deep
neural network as presented in this work.
