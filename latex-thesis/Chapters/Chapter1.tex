
% Chapter 1

\chapter{Introduction} % Main chapter title

\label{Chapter1} % For referencing the chapter elsewhere, use \ref{Chapter1}

% This is for the header on each page - perhaps a shortened title
\lhead{Chapter 1. \emph{Introduction}}

% Quotation
{``As a reader I loathe introductions...Introductions inhibit pleasure, they kill the joy of
anticipation, they frustrate curiosity."}
\begin{flushright}
Harper Lee, \textit{To Kill a Mockingbird} (1960)
\end{flushright}

Road traffic congestion is a serious global issue, resulting in significant wastage of resources.
While improving and extending the road infrastructure has reduced the issue to some extent, this is
time consuming and limited in capacity. Thus in last few decades, for better planning and control
of road traffic, adaptive traffic control systems have been deployed around the world. Still the
role of adaptive control systems is not fully realised without predictive capabilities in the
short term, without which the adapative traffic control systems only react to events at the real
time. While this is the desired objective, the performance of these systems could be significantly
improved by making them proactive (\citet{smith1997traffic}). Short term traffic flow prediction
is not only helpful for these advanced traffic control systems, it is also useful for advanced
traveller information systems.

\section{Background}
Predicting the future has always been a fascinating topic throughout the history of mankind.
Instances of predicting the future through unconventinal means have been mentioned in various
forms of literature such as mythologies, fantasy, science fiction etc. Even today, while the
means of prediction have changed, we still try to predict almost everything in our day to day
lives - from election polls to sports outcomes to financial results.

In the context of road traffic, prediction has always been a difficult task, merely due to the
dynamic and complex nature of traffic. A lot of research has gone into short term
traffic prediction. Various methods have been proposed in last few decades to provide more
accurate predictions, yet none of those methods has been claimed to become the best.

\section{Objectives and scope}

\textbf{Research objective} is to propose a new model that can use the large amount of available
traffic data to predict the traffic in the short term. More importantly this research tries to
answer the following questions -

\begin{itemize}
\item How can we use the large amount of traffic data available for predicting short term traffic
 flow?
\item Can the proposed deep learning models have better accuracy than exiting models?
\end{itemize}

\textbf{Research scope} - The scope of this research is to predict traffic volume in the
short term. The use of the phrase 'short term' implies that we are only interested in the
prediction within a very short horizon which typically ranges between few seconds to few
hours in practice. The traffic parameters that are of usually of interest to be predicted are
volume, time, speed and density. The scope of this research is limited to the prediction of only
traffic volume. While doing so, we are only taking the past data into consideration and not
taking the non-recurrent phenomena such as traffic accidents, weather or public events into
consideration.


\section{Thesis outline}

\textbf{Chapters} -- This thesis is divided into six chapters.

\begin{itemize}
\item Chapter 1: Introduction - In this chapter we present the background and research context,
research objectives and scope.

\item Chapter 2: Traffic Prediction: Literature Review - In this chapter we provide a reasonably
thorough review of existing literature on short term traffic prediction.

\item Chapter 3: SCATS Traffic Volume Data - In this chapter we provide the description of the
traffic volume data collected by VicRoads using the SCATS systems. Methods to deal with missing
data are presented in this chapter. Finally we present some exploratory data analysis on the
traffic data.

\item Chapter 4: A Deep LSTM Network for Short Term Traffic Prediction - In this chapter we propose
a long short term memory (LSTM) neural network model for short term traffic prediction.

\item Chapter 5: Evaluation of the Model - In this chapter we evaluate the model.

\item Chapter 6: Conclusions and Future Directions - In this chapter we conclude our thesis and
provide inputs for future work.
\end{itemize}
