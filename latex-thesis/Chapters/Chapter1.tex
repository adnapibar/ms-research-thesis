
% Chapter 1

\chapter{Introduction} % Main chapter title

\label{Chapter1} % For referencing the chapter elsewhere, use \ref{Chapter1}

% This is for the header on each page - perhaps a shortened title
\lhead{Chapter 1. \emph{Introduction}}

% Quotation
{``As a reader I loathe introductions...Introductions inhibit pleasure, they kill the joy of
anticipation, they frustrate curiosity."}
\begin{flushright}
Harper Lee, \textit{To Kill a Mockingbird} (1960)
\end{flushright}

\section{Background}
Predicting the future has always been a fascinating topic throughout the history of mankind.
Instances of predicting the future through unconventional means have been mentioned in various
forms of literature such as mythologies, fantasies, science fictions etc. Even today, while the
means of prediction have changed, we still try to predict almost everything in our day to day
lives - from election polls to sports outcomes to financial results.

Road traffic congestion is a serious global issue, resulting in significant wastage of time and
resources. Several factors such as growth in population, urbanisation and affordable personal vehicles
have aggravated the issue. In Australia the number of personal vehicles have grown from 1.4 million
to 13 million during the period from 1955 to 2013, an average annual growth of 4\%\footnote{Australian
Bureau of Statistics - \url{http://www.abs.gov.au/AUSSTATS/abs@.nsf/Lookup/4102.0Main+Features40July+2013}}.
In 2012, the majority of Australians used personal vehicles, 72\% to work or study and 88\% for other activities.
When it comes to travel time, across Sydney and Melbourne, the overall travel time is 37\% and 29\%
more than the normal because of traffic
congestions\footnote{TomTom Traffic Index - \url{http://www.tomtom.com/en_au/trafficindex/list}}.
Across the globe, this figure is far worse in many other cities like Mexico City (59\%), Bangkok (57\%),
Istanbul (50\%), Rio de Janiro (47\%) and Moscow (44\%). While improving and extending the road infrastructure has
reduced the issue to some extent, this is time consuming and does not eliminate the issue.
Thus in last few decades, for better planning and control of road traffic, advanced traffic
management systems have been deployed around the world. Still the role of these systems is not
fully realised without predictive capabilities in the short term, without which these systems only
react to events at real time. While this is the desired objective, the performance of these systems
could be significantly improved by making them proactive (\citet{smith1997traffic}). Short term traffic
flow prediction is not only helpful for these advanced traffic control systems, it is also useful
for advanced traveller information systems.


Research in short term traffic prediction had been active for more than three decades. This shows the
strong interest in solving the growing problem of traffic congestion by providing accurate predictions
that can be used in both advanced traffic management systems and advanced traveller information systems.
However due to the complex nature of traffic conditions, a globally applicable short term prediction
model that can easily be embedded into advanced traffic management and advanced traveller information
systems is yet to be found.

Neural networks have the potential to solve such complex problems, and it has been realised long term
ago. But due to the practical issue that arise in training, their use have been restricted. In last
decade new ways of training deep neural networks have made them reemerged as victors. Also the use
of fast graphical processing units have reduced the training time by 10 to 20 times. The capabilities
that neural networks had initially promised have been realised in several fields such as machine
vision, speech recognitions and language modelling. In this study we extend their
applications to the field of short term traffic predictions. We used several variants of deep recurrent
neural networks to model the traffic conditions and make predictions
at multiple steps ahead at a wider multi-location level. Our study shows some promising results and encourages
similar future works.


\section{Motivations}
There are two main factors that have motivated us to undertake this study. These are

\begin{itemize}
\item Short term traffic prediction has more than three decades of active research, making this a
challenging task. This is mainly due to the complex nature of traffic data - temporal and spatial
relationships, noise and missing values, effects of non-recurrent events (weather, accidents,
public events etc.)

\item Availability of huge amount of historical data and new breakthroughs in deep learning.

\end{itemize}

\section{Objectives and scope}

\textbf{Research objective} is to use the large amount of available historical traffic volume data and
apply deep neural networks for future predictions at a wider multi-location level. More importantly this research
tries to answer the following questions -

\begin{itemize}
\item How can we use the large amount of traffic data available for predicting short term traffic
 flow for better predictions?
\item Can we use the ability of deep neural networks to capture the tempo-spatial dimension of
traffic conditions and make predictions at a transportation network level?
\end{itemize}

\textbf{Research scope} - The scope of this research is to predict traffic volume in the
short term. The use of the phrase 'short term' implies that we are only interested in the
predictions within a very short time horizon which typically ranges between few seconds to few
hours in practice. The traffic parameters that are of usually of interest to be predicted are
volume, time, speed and density. The scope of this research is limited to the prediction of only
traffic volume. While doing so, we are only taking the past data into consideration and not
taking the non-recurrent phenomena such as traffic accidents, weather or public events into
consideration.


\section{Thesis outline}

\textbf{Chapters} -- This thesis is divided into six chapters.

\begin{itemize}
\item Chapter 1: Introduction - In this chapter we present the background and research context,
research objectives and scope.

\item Chapter 2: Traffic Prediction: Literature Review - In this chapter we provide a reasonably
thorough review of existing literature on short term traffic prediction.

\item Chapter 3: SCATS Traffic Volume Data - In this chapter we describe the traffic volume data
collected by VicRoads using the SCATS systems. Methods to deal with missing data are presented in
this chapter. Finally we present some exploratory data analysis on the traffic data.

\item Chapter 4: Deep Neural Networks for Short Term Traffic Prediction - In this chapter we present
the details of deep neural networks with emphasis on Long Short Term Memory (LSTM) neural network.

\item Chapter 5: Experiments and Results - In this chapter we conduct experiments using several
existing methods in short term traffic prediction and three variants of deep recurrent neural
networks. The results of these experiments are presented. Three accuracy measures were used to
draw a comparison among the models.

\item Chapter 6: Conclusions and Future Directions - In this chapter we conclude our thesis and
provide inputs for future work.
\end{itemize}
