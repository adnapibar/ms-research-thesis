% Chapter 2

% Main chapter title
\chapter{Traffic Prediction: Literature Review}

% For referencing the chapter elsewhere, use \ref{Chapter2}
\label{Chapter2}

% This is for the header on each page - perhaps a shortened title
\lhead{Chapter 2. \emph{Traffic Prediction: Literature Review}}

% Quotation
{``There is no way that we can predict the weather six months ahead beyond giving the seasonal
average"}
\begin{flushright}
Stephen Hawking, \textit{Black Holes and Baby Universes} (1993)
\end{flushright}

%---------------------------------------------------------------------------------------------------
%	CONTENT
%---------------------------------------------------------------------------------------------------
\section{Introduction}
The goal of this chapter is to provide a thorough review of prior work in the area of traffic flow
prediction. In this context we are only interested in the short term traffic forecasting where the
duration is of few minutes. The various methods that have been applied in the area of traffic
prediction can be in general categorised as parametric and nonparametric methods. The parametric
metric methods include linear and non-linear regression, Kalman filter and ARIMA and its
variants. The nonparametric methods include nonparametric regression, support vector machines and neural networks.

\section{Parametric models}
In parametric models, estimation of parameters are

\subsection{Linear and non-linear regression}
~\citet{hogberg1976estimation} proposed applying a non-linear regression model in prediction traffic flow.

\subsection{ARIMA}
% A brief introduction to time series, ARMA and ARIMA models
ARIMA(Auto Regressive Integrated Moving Average) is a class of parametric regression model

For an in depth understanding of these models the reader is encouraged to refer to to
~\citet{tong1990non}, ~\citet{brockwell2006introduction} and ~\citet{box2015time}.


%ARIMA and its variants in traffic forecast
\citet{ahmed1979analysis} used Box-Jenkins method for short-term traffic forecast. The input data
used was 166 sets of time series traffic data collected by freeway traffic surveillance systems in
three locations - Los Angeles, Minneapolis and Detroit. The authors concluded an ARIMA(0,1,3) model
as a resonable fit for the short term prediction task.

\citet{kumar2015short} used a seasonal ARIMA in a context of limited data for short term traffic prediction.

\subsection{Kalman filter}

\subsection{Other parametric models}

\section{Nonparametric models}
In nonparamtric models the parameters are not fixed, and vary with the amount of data available.
Usually more data is required for this models than parametric models. The advantage of these models
is that they can model the complex non-linear data better. Some of the widely used nonparametric
models are - k-Nearest Neighbour, Non-parametric regrssion and Neural Networks

\subsection{k-Nearest neighbour}

\subsection{Nonparametric regression}

\subsection{Support vector machines}

\subsection{Neural networks}
Neural network is modelled on how human brain performs a particular task. We can view a neural
network as a parallel distributed network made up of simple processing units. The networks acquire
knowledge through the process of learning and this knowledge is stored as interneuron connection
strengths known as synaptic weights.

\section{Other Methods}

\subsection{Knowldge Based Systems}

\subsection{Hybrid Methods}

\section{Comparisons}
