% Chapter 2

% Main chapter title
\chapter{Traffic Prediction: Literature Review}

% For referencing the chapter elsewhere, use \ref{Chapter2}
\label{Chapter2}

% This is for the header on each page - perhaps a shortened title
\lhead{Chapter 2. \emph{Traffic Prediction: Literature Review}}

% Quotation
{``There is no way that we can predict the weather six months ahead beyond giving the seasonal
average"}
\begin{flushright}
Stephen Hawking, \textit{Black Holes and Baby Universes} (1993)
\end{flushright}

%---------------------------------------------------------------------------------------------------
%	CONTENT
%---------------------------------------------------------------------------------------------------
\section{Introduction}
This objective of this chapter is to provide a reasonably complete review of existing literature on
short term traffic flow prediction. In the context of traffic congestion the the forecasting window
of traffic flow is very important.

After three decades of intensive research, short term traffic flow prediction is still a subject of
interest for many professionals around the world.  The various methods that have been applied in
the area of traffic prediction can be in general categorised as parametric and nonparametric
methods. The parametric metric methods include linear and non-linear regression, Kalman filter and
ARIMA and its variants. The nonparametric methods include nonparametric regression, support vector
machines and neural networks.

\section{Parametric models}
In parametric models, we estimate the parameters from the training dataset to determine the
function that classifies new unseen data. The number of parameters are fixed. The advantage of
parametric models are that these perform quite well in situations where the large amount of data
is not available. Some of the typical examples of parametric models include time series
regression, ARIMA models, Kalman filter, linear SVM etc.

\subsection{Linear regression}
In machine learning and statistical applications, the use of linear models are predominant. These
models are also important in time series domains such as traffic flow prediction. The primary
idea behind the regression is to express the output variable as a linear combination of input
vectors. We can express the linear regression in time series as an ouput influenced by a
collection of inputs, where the inputs could possibly be an independent series

        \begin{equation}
            x_{t} = \beta_{1}z_{t1} + \beta_{2}z_{t2} + ... + \beta_{q}z_{tq} + w_{t}
        \end{equation}

where $ \beta_{1}, \beta_{2},...,\beta_{q} $ are unknown regression coeffiecients and $w_{t}$ is
a random error.

\subsection{ARIMA}
%  Introduction to ARIMA models
ARIMA(Auto Regressive Integrated Moving Average) is a class of parametric regression models. In
this section we will introduce ARIMA and related methods such as exponential smoothing and moving
averages. For an in depth understanding of these models the reader is encouraged to refer to to
~\citet{tong1990non}, ~\citet{brockwell2006introduction} and ~\citet{box2015time}.

The main idea behind autoregressive models is that past values affect the present value, i.e.
$x_{t}$ can be expressed as a function of past p values $ x_{t-1}, x_{t-2},...,x_{t-p} $ , where
p is the number of steps into the past. We can express an autoregressive model of order p as below

        \begin{equation}
          x_{t} = \alpha_{1}x_{t-1} + \alpha_{2}x_{t-2} + ... + \alpha_{p}x_{t-p} + w_{t}
        \end{equation}

where $x_{t}$ is stationary and $ \alpha_{1}, \alpha_{2},..., \alpha_{p} $ are constants. We have
added the term $w_{t}$ as a Guassian white noise.


% Application in traffic forecast
\citet{ahmed1979analysis} used Box-Jenkins method for short-term traffic forecast. The input data
used was 166 sets of time series traffic data collected by freeway traffic surveillance systems in
three locations - Los Angeles, Minneapolis and Detroit. The authors concluded an ARIMA(0,1,3) model
as a resonable fit for the short term prediction task.

\citet{kumar2015short} used a seasonal ARIMA in a context of limited data for short term traffic 
prediction.

\subsection{Kalman filter}

\subsection{Other parametric models}

\section{Nonparametric models}
In nonparamtric models the parameters are not fixed, and vary with the amount of data available.
Usually more data is required for this models than parametric models. The advantage of these models
is that they can model the complex non-linear data better. Some of the widely used nonparametric
models are - k-Nearest Neighbour, Non-parametric regrssion and Neural Networks

\subsection{k-Nearest neighbour}

The basic process of the k-nearest neighbour algorithm can be described as in the figure ~
\ref{fig:KnnProcessFlow}

\tikzstyle{block} = [rectangle, draw, text width=5em, text centered, rounded corners, minimum height=4em]
\tikzstyle{line} = [draw, -latex']
\tikzstyle{cloud} = [draw, ellipse, node distance=4cm, minimum height=3em]

\begin{figure}
\centering
\begin{tikzpicture}[node distance = 3cm, auto]
    % Place nodes
    \node [block] (pp) {Preprocess Data};
    \node [cloud, left of=pp] (hd) {Historical data};
    \node [cloud, right of=pp] (rd) {Real-time data};
    \node [block, below of=pp] (msv) {Match state vector};
    \node [block, below of=msv] (knn) {K nearest neighbour};
    \node [block, left of=knn, node distance=3cm] (prd) {Predictions};
    % Draw edges
    \path [line] (pp) -- (msv);
    \path [line] (msv) -- (knn);
    \path [line] (knn) -- (prd);
    \path [line,dashed] (hd) -- (pp);
    \path [line,dashed] (rd) -- (pp);
\end{tikzpicture}
\caption{K nearest neighbour process flow} \label{fig:KnnProcessFlow}
\end{figure}

\subsection{Nonparametric regression}

\subsection{Support vector machines}

\subsection{Neural networks}

\subsection{Other nonparametric models}

\section{Other Methods}

\subsection{Knowldge Based Systems}

\subsection{Hybrid Methods}

\section{Comparisons}
